% Исходный LaTeX-код (c) Пётр Калинин
% Код распространяется по лицензии GNU GPL (!)

\header{Способы написания ДП}
Если вы вывели рекуррентное соотношение, то вы фактически решили задачу. Но остаётся ещё несколько
технических моментов, которые надо уметь обрабатывать. Один из них мы уже обсуждали "--- это вывод
ответа по насчитанной матрице. Тут мы обсудим немного другой момент "--- как собственно можно писать
вычисление значений матрицы.

\lheader{ДП с просмотром назад} 
Вы можете спросить: зачем это все, ведь мы уже видели, что все просто. Если есть рекуррентное
соотношение, то просто вычисляем все значения в матрице в цикле или нескольких вложенных циклах "---
и все!

Да, в большинстве случаев все действительно так просто. Это и называется ДП с просмотром назад: вы в цикле перебираете все подзадачи, дойдя до очередной подзадачи, вы вычисляете ответ на неё, используя ответы на более простые подзадачи, которые вы вычислили раньше. Тут полезно представить себе орграф подзадач: вы перебираете вершины одну за другой и, встав в очередную вершину, смотрите, \textit{куда} из неё идут ребра. Увидели, куда, посмотрели, какие там уже насчитаны ответы, и собрали из этих ответов с помощью рекуррентного соотношения ответ на свою подзадачу.

Все просто, и в большинстве случаев вы именно так и будете писать динамику. Но есть ещё два способа написания динамики, у которых есть определённые преимущества. Читайте далее.

\lheader{ДП с просмотром вперёд}  Это "--- довольно хитрая идея, я её сначала проиллюстрирую на примере задачи про 01"=последовательности. Рассмотрим такой код:
\begin{codesampleo}\begin{verbatim}
fillchar(ans,sizeof(ans),0);
ans[1]:=2;ans[2]:=1;
for i:=1 to n do begin
    ans[i+1]:=ans[i+1]+ans[i];
    ans[i+2]:=ans[i+2]+ans[i];
end;
\end{verbatim}
\end{codesampleo}
Утверждается, что этот код корректно решает задачу о 01"=последовательностях. С ходу это может показаться неочевидным, но работает это так. Мы перебираем все подзадачи и, оказавшись в очередной вершине графа подзадач, смотрим, какие вершины зависят \textit{от неё}, т.е. \textit{откуда} в неё идут ребра. Мы знаем, что ответ на каждую подзадачу "--- это сумма ответов на все подзадачи, от которых она зависит. Поэтому мы просто для каждой подзадачи, которая зависит \textit{от нашей}, добавляем к её ответу наш ответ. Поэтому "--- главная идея всего этого! "--- \textit{когда мы доберёмся од очередной подзадачи, в соответствующей ячейке таблицы уже будет храниться правильный ответ на неё!} Ещё раз: ответ на задачу с $i=10$ равен сумме ответов на $i=9$ и $i=8$. Изначально $ans[10]=0$. На \textit{восьмой} итерации цикла, т.е. при $i=8$, второй строчкой цикла мы выполним присваивание $ans[10]:=ans[10]+ans[8]$, в результате чего станет $ans[10]=ans[8]$. На \textit{девятой} итерации цикла (при $i=9$) мы сделаем $ans[10]:=ans[10]+ans[9]$, в результате чего $ans[10]$ станет равным $ans[9]+ans[8]$, что и требовалось! Поэтому на \textit{десятой} итерации цикла нам уже ничего не надо делать для вычисления $ans[10]$: он у нас вычислен!

Если ещё не понятно, то, может быть, все станет понятнее, если промоделировать работу этого цикла:
\begin{tabular}{r|>{\quad\bfseries}c>{\quad\bfseries}c>{\quad\bfseries}c>{\quad\bfseries}c>{\quad\bfseries}cc}
изначальное состояние массива&2 &1 &0 &0 &0\\
\multicolumn{7}{c}{первая итерация цикла: $i=1$: $ans[2]:=ans[2]+ans[1]$; $ans[3]:=ans[3]+ans[1]$}\\
после первой итерации&2 &3 &2 &0 &0\\
\multicolumn{7}{c}{вторая итерация цикла: $i=2$: $ans[3]:=ans[3]+ans[2]$; $ans[4]:=ans[4]+ans[2]$}\\
после второй итерации&2 &3 &5 &3 &0\\
\multicolumn{7}{c}{третья итерация цикла: $i=3$: $ans[4]:=ans[4]+ans[3]$; $ans[5]:=ans[5]+ans[3]$}\\
после третьей итерации&2 &3 &5 &8 &5\\
\multicolumn{7}{c}{четвёртая итерация цикла: $i=4$: $ans[5]:=ans[5]+ans[4]$; $ans[6]:=ans[6]+ans[4]$}\\
после четвёртой итерации&2 &3 &5 &8 &13\\
\multicolumn{7}{c}{и т.д.}\\
\end{tabular}

Я надеюсь, теперь совсем понятно; по крайней мере видно, что значения получаются правильные.

Конечно, здесь понадобится соответствующее расширение массива, чтобы не было RE на последних итерациях, но это уже довольно очевидно.

В чем достоинство такого способа? Основное достоинство в том, что вы ходите по рёбрам графа не в направлении стрелок, а в обратном направлении. Иногда так бывает (я знаю одну такую задачу), когда перебрать задачи, \textit{зависящие от данной}, легко, а вот перебрать задачи, \textit{от которых зависит данная}, нетривиально "--- в таком случае ДП с просмотром вперёд написать проще.

Ещё обращу внимание, что (как мне это видится) не любое рекуррентное соотношение можно использовать для ДП с просмотром вперёд. Если ответ на подзадачу является суммой ответов на мелкие подзадачи, или минимумом из них, или минимумом, к которому прибавлена некоторая константа, или квадратом максимума, и т.п., то все просто. В более общем случае использование ДП с просмотром вперёд может быть осложнено.

Пример: задача про Буратино (не знаю, почему она так называется, но так принято). 

Папа Карло сменил работу: теперь он работает в мастерской, и целый рабочий день занимается тем, что
забивает гвоздики. Чтобы ему было не скучно, у него в мастерской стоит постоянно работающий
телевизор. К сожалению, производительность папы Карло напрямую зависит от его настроения, а оно, в
свою очередь, "--- от того, что в данный момент показывают по телевизору. Правда, пока папа Карло
забивает гвоздик, он не обращает ни малейшего внимания на телевизор, и поэтому скорость его работы
зависит только от того, что показывали по телевизору в тот момент, когда он только начал забивать
этот гвоздик. Забив очередной гвоздик, он обязательно мельком смотрит в телевизор (его настроение,
естественно, меняется), и после этого он может либо сразу начать забивать следующий гвоздик, либо
отдохнуть несколько секунд или даже минут, смотря телевизор.

Известна программа телевизионных передач и то, как они влияют на папу Карло. Требуется составить
график работы и маленьких перерывчиков папы Карло так, чтобы за рабочий день он вбил максимально
возможное количество гвоздей.

Во входном файле записано расписание телевизионных передач с 9:00:00 до 18:00:00 в следующем
формате. В первой строке число $N$ "--- количество телевизионных передач в этот период ($1\leq N\leq
32400$). В каждой из последующих $N$ строк записано описание одной передачи: сначала время её начала
в формате ЧЧ:ММ:СС (ЧЧ "--- две цифры, задающие часы, ММ "--- две цифры, задающие минуты начала, СС
"--- две цифры, задающие секунды начала). А затем через один или несколько пробелов число $T_i$ "---
время в секундах, которое папа Карло будет тратить на забивание одного гвоздика, если он перед этим
увидит по телевизору эту передачу ($1\leq T_i\leq 32400$).

Передачи записаны в хронологическом порядке. Первая передача всегда начинается в 09:00:00. Можно
считать, что последняя передача заканчивается в 18:00:00.

(Конец условия задачи)

Не обращайте внимание на технические проблемы в этой задаче (хитрый формат ввода, тонкости с тем,
что будет, если он не успеет дозабивать последний гвоздик и т.п.). Обратите внимание на то, что
всего секунд с 9:00:00 до 18:00:00 не так уж и много: всего 32400, и никто не мешает вам выделить
массив такой длины, и работать за $O(\mbox{\it количества секунд})$.

\task|Решите эту задачу.
||Идея простая: давайте для каждого момента времени $t$ посчитаем, сколько максимум гвоздиков может 
забить папа Карло к моменту времени $t$. Как решить подзадачу? Наученные опытом прошлых задач (и 
задачи \ref{multi_coins}, и концом раздела \ref{subsequence}, если вы до туда добрались, и другими 
задачами),
мы не будем перебирать, когда папа Карло закончит забивать последний гвоздь, а просто рассмотрим 
два варианта: либо папа Карло закончил забивать последний гвоздь в момент $t$, либо раньше "--- и 
тогда от $t-1$ до $t$ у него был перерывчик. 

Все просто, но рассмотреть первый вариант 
нетривиально: когда папа Карло мог начать забивать гвоздик, если закончил в момент $t$? Перебирать 
все вообще моменты времени долго, и сложность получится квадратичной, но можно проще, 
воспользовавшись идеей динамики с просмотром вперёд.
||Итак, заметим, что найти, \textit{от каких задач} зависит задача $t$, очень нетривиально: либо
перебрать все предыдущие моменты $t'$ и посмотреть, подходит ли момент $t'$ нам (т.е. верно ли, что 
$t'+time[t']=t$, где $time[t']$ "--- сколько секунд папа Карло будет забивать гвоздик, начав в 
момент $t'$), либо построить полный граф подзадач, пробежавшись заранее по всем моментам $t'$ и для 
каждого посчитав $t'+time[t']$ и добавив момент времени $t'$ в связный список, хранящий подзадачи 
для момента времени $t'+time[t']$\dots

Но, с другой стороны, очень просто понять, какие задачи зависят \textit{от} задачи $t$. Поэтому 
пишем ДП с просмотром вперёд. Ещё раз: от задачи $t$ зависят две задачи: $t+1$ (если делать 
маленький перерывчик) и $t+time[t]$ (если начать забивать гвоздик). Получаем код:
\begin{codesampleo}\begin{verbatim}
fillchar(ans,sizeof(ans),0);
ans[0]:=0;
for i:=1 to n do begin
    ans[i+1]:=max(ans[i+1],ans[i]); 
    ans[i+time[i]]:=max(ans[i+time[i]],ans[i]+1);
end;
\end{verbatim}
\end{codesampleo}
Все!

И напоследок замечу, что эту задачу, видимо, можно легко решить без всякого просмотра вперёд, 
"<обратив"> динамику и для каждого момента времени $t$ вычисляя, сколько максимум гвоздей сможет 
папа Карло забить от момента $t$ до конца рабочего дня. Додумайте. Может быть, такой вариант даже 
возможен во всех задачах на ДП с просмотром вперёд; не знаю. Тем не менее это не повод пренебрегать 
просмотром вперёд :)
|\label{buratino}

\lheader{Рекурсия с запоминанием результата} Оно же ленивое ДП.
К этой идее можно придти разными способами, попробую изложить все.

Вообще, зачем нам что-то новое? Мы вроде и так умеем писать ДП, даже зачем"=то выучили ДП с 
просмотром вперёд? Но если подумать, то в обоих рассмотренных выше способах написания ДП есть два 
недостатка. Первый состоит в том, что нам надо заранее определить, в каком порядке мы будем решать 
подзадачи, чтобы, когда мы дойдём до очередной подзадачи, все задачи, от которых она зависит, уже 
были бы обработаны. В простых случаях найти такой порядок не представляет сложностей, но возможны 
случаи, когда все не так очевидно. Как же найти такой порядок в общем случае? А очевидно. Нам надо 
упорядочить подзадачи так, чтобы все ребра в графе подзадач шли от более поздних подзадач к более 
ранним "--- а ведь это топологическая сортировка, которую мы уже прекрасно знаем!

Есть и другой недостаток у простой реализации ДП. Мы выше всегда вычисляли ответы на каждую подзадачу, при том, что, возможно, не все эти ответы мы будем когда"=нибудь использовать. В задаче про черепашку и про 01"=последовательности такого эффекта нет, но в задаче про монеты несложно видеть, что нам обычно не обязательно решать \textit{каждую} подзадачу. Например, там совершенно незачем выяснять, можно ли \textit{всеми} монетами набрать какую"=нибудь сумму, отличную от той, что дана во входном файле. Если все монеты невелики, а сумма достаточно большая, то ясно, что нам не надо смотреть, можем ли мы набрать небольшие суммы почти всеми монетами; если все монеты чётны, то заранее ясно, что нечётные суммы набрать не получится "--- короче, ясно, что не всегда надо решать все подзадачи. Более того, ясно, что можно придумать много критериев, какие подзадачи \textit{не} надо решать, но все подобные критерии будут не очень тривиальны, существенно зависеть от входного файла и т.д. "--- в общем, нужен другой подход.

А этот другой подход довольно очевиден, если опять вспомнить про граф подзадач. Мы знаем, какую подзадачу надо точно решить "--- ту, ответ на которую надо вывести в выходной файл "--- и, зная граф подзадач, легко можем определить, какие именно надо решать "--- просто поиском в глубину из конечной вершины! При этом мы сразу и без проблем точно определим минимальное количество задач, которые надо решить, и будем решать только их.

Наконец, посмотрим на ДП"=задачи совсем с другой стороны. Я уже обращал ваше внимание на аналогию между ДП и перебором. Ещё раз повторю то же, но чуть"=чуть по"=другому. Пусть мы вывели рекуррентное соотношение. Тогда, казалось бы, мы просто пишем функцию, которая вычисляет ответ на подзадачу: она будет строго следовать рекуррентному соотношению, для определения входящих в это соотношение ответов на более мелкие подзадачи будет использовать, естественно, рекуррентный вызов. Для задачи про монеты получаем код
\begin{codesampleo}\begin{verbatim}
function find(i,j:integer):boolean;
begin
if i=0 then begin
   find:=j=0; {понимаете такую конструкцию?}
   exit;
end;
if j<a[i] then
   find:=find(i-1,j)
else find:=find(i-1,j-a[i]) or find(i-1,j);
end;
\end{verbatim}
\end{codesampleo}
(сравните код с рекуррентным соотношением, приведённым в разделе \ref{coins}). Этот код, конечно, работает, но мы уже видели, что он работает медленно, потому что по много раз вычисляет ответы на одну и ту же задачу. И тут в голову сразу приходит мысль: а давайте будем в отдельном массиве запоминать, какие задачи мы уже решили, а какие нет, и пусть функция, прежде чем что"=либо делать, посмотрит в этот массив и проверит, не решали ли мы раньше эту задачу.

Итак, у нас будет массив $ans[i,j]$, элементы которого будут иметь следующий смысл: если $ans[i,j]=-1$, то эту подзадачу мы ещё не решали; $ans[i,j]=0$ "--- решали и ответ "<нельзя"> (набрать сумму $j$, используя первые $i$ монет), $ans[i,j]=1$ "--- решали и ответ "<можно">. Тогда получаем код

\begin{codesampleo}\begin{verbatim}
function find(i,j:integer):boolean;
begin
if i=0 then begin
   find:=j=0;
   exit;
end;
if ans[i,j]=-1 then begin
  if j<a[i] then
    ans[i,j]:=find(i-1,j)
  else begin
    if (find(i-1,j-a[i])=1) or (find(i-1,j)=1) then
      ans[i,j]:=1
    else ans[i,j]:=0;
  end;
end;
find:=ans[i,j];
end;
\end{verbatim}
\end{codesampleo}
(Такой страшный if просто чтобы проще понимать было.)

Все! Теперь такая рекурсия работает столь же быстро, что и обычное ДП с просмотром назад, т.к. каждая задача решается только один раз. Более того, такое написание очевидно решат обе указанные в начале этого параграфа проблемы: проблему с определением порядка решений подзадач и проблему с тем, какие именно подзадачи надо решать.

Действительно, этот код как раз и реализует поиск в глубину в графе подзадач (очевидно?), и решает только те задачи, до которых дошёл. Поэтому он решает только те задачи, которые нам на самом деле нужны, делая как раз то, о чем мы говорили выше.

А по поводу определения порядка обработки подзадач, мы договорились до того, что надо оттопсортить граф подзадач. А для топсорченья надо, как мы знаем, просто запустить поиск в глубину и на выходе из процедуры поиска занести вершину в выходной массив, а потом пробежаться по этому массиву слева направо и решить все подзадачи. Но зачем нам тут выходной массив? Мы ведь заносим в него вершины в том порядке, в котором их потом, при ДП"=вычислениях, будем обрабатывать "--- а тогда давайте сразу на выходе из процедуры поиска в глубину обработаем эту вершину графа подзадач, т.е. просто посчитаем ответ на эту подзадачу. Теперь, если вдуматься, то ясно, что приведённый выше код как раз и реализует топсорт графа подзадач с вычислением ответов на подзадачи при выходе из процедуры поиска в глубину.

Кстати, помните аргументацию к топсорту? "<Процедура $put$ ставит вершину $i$ в выходной массив. Прежде чем туда её поставить, она пытается поставить туда все вершины, которые должны идти после $i$-ой (напомню, что массив мы заполняем с конца); естественно, это делается рекурсивным вызовом. После того, как это выполнено, 
можно непосредственно поместить $i$ в выходной массив."> (Там, правда, мы хотели получить порядок такой, чтобы все ребра шли слева направо, а сейчас нам надо справа налево, поэтому и заполняли массив с конца и поэтому вершины, куда идут ребра из $i$"=ой, должны были идти после неё, а сейчас нам надо наоборот.) Аргументация у нашего алгоритма абсолютно аналогичная: процедура $find$ вычисляет ответ для вершины. Прежде чем его вычислить, она пытается вычислить ответ для всех вершин, от которых зависит $i$"=ая. После того, как это выполнено, можно непосредственно вычислить ответ для $i$"=ой вершины. Т.е. ещё раз: то, что мы написали и топсорт "--- это, фактически, одно и то же, и идеология у них одна и та же.

Итак, это и называется рекурсией с запоминанием результата, или ленивым ДП (вообще, ленивым, насколько я понимаю, называется что угодно, что делает то или иное действие только тогда, когда оно нам действительно понадобится "--- так и здесь, мы вычисляем очередной ответ, только когда он нам стал очень нужен). В обычных задачах его писать немного более громоздко, чем обычное ДП (хотя, может быть, понять проще), но есть класс задач (динамика на деревьях или ДП на ациклических графах), когда рекурсия с запоминанием результата "--- самый естественный способ написания ДП. Такие задачи мы ещё обсудим ниже.

