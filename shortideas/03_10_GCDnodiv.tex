% Исходный LaTeX-код (c) Пётр Калинин
% Код распространяется по лицензии GNU GPL (!)

{\newcommand{\GCD}{\mbox{НОД}}
\lheader{НОД без деления} 
Я надеюсь, вы все представляете, как считать НОД двух чисел алгоритмом Евклида. На всякий случай 
напомню: пусть вам надо посчитать $\GCD(a,b)$. Если $b=0$, но $\GCD(a,b)=a$, иначе $\GCD(a,b)=\GCD(b, a \bmod b)$.
Получаем следующий алгоритм:

\begin{codesampleo}\begin{verbatim}function gcd(a,b:integer):integer;
begin
if b=0 then
   gcd:=a
else gcd:=gcd(b,a mod b);
end;\end{verbatim}
\end{codesampleo}

Можно показать, что этот алгоритм работает за $O(\max(\log a,\log b))$. В частности, поэтому имеет 
смысл поставить задачу поиска НОД \textit{длинных} чисел. Но, как только вы глянете на приведённый 
выше код, сразу поймёте, что написать такое будет весьма нетривиально и неприятно. Действительно, 
кому охота реализовывать деление длинного на длинное (для mod)?

Но есть способ посчитать НОД без деления. Точнее, без деления длинного на длинное. А именно, 
применим ту же идею, что применяли для быстрого возведения в степень. Посмотрим на остатки $a$ и 
$b$ по модулю 2. Возможны четыре случая:
\begin{ulist}
\item $a \bmod 2=b\bmod 2=0$. Тогда очевидно, что $\GCD(a,b)=2\cdot\GCD(a/2,b/2)$.
\item $a\bmod 2=0$, но $b\bmod 2=1$. Тогда $\GCD(a,b)=\GCD(a/2,b)$.
\item $a\bmod 2=1$, и $b\bmod 2=0$. Догадайтесь сами :)
\item $a\bmod 2=b\bmod 2=1$. Тогда $\GCD(a,b)=\GCD(|a-b|,b)$ типа того (модуль для того, чтобы 
числа оставались положительными; может быть, можно и без него. Можно обойти эту проблему и как"=нибудь по"=другому...).
\end{ulist}

\task|Докажите все эти утверждения.|||||

Этот алгоритм тоже работает за $O(\max(\log a,\log b))$, т.к. каждый вариант, кроме последнего, 
уменьшает хотя бы одно из чисел в 2 раза, и не бывает так, чтобы два раза подряд получился 
последний вариант (почему? :)), поэтому 
как минимум каждая вторая итерация будет уменьшать хотя бы одно из чисел в два раза. 

Но главное "--- здесь надо уметь делить только на два. Надеюсь, написать длинное деление и 
вычисление остатка по модулю два для вас элементарно :).

}%newcommand
