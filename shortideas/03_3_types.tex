% Исходный LaTeX-код (c) Пётр Калинин
% Код распространяется по лицензии GNU GPL (!)

\lheader{Нетривиальные типы данных в паскале}

\textbf{Вещественные типы.} Как известно, в паскале есть 4 "<настоящих"> вещественных типа: 
real, single, double и extended (про comp разговор будет ниже). Из них последние три "--- 
стандартные типы, они поддерживаются математическим сопроцессором, который, насколько я понимаю, 
давно встроен во все современные  процессоры. А первый (real) в Borland Pascal обрабатывался 
полностью программно, из процессорных функций используя, 
видимо, только работу с целыми числами; в современных версиях это как правило или single или 
double, в зависимости от настроек (а, возможно, при определенных настройках это может быть и старый странный тип).  

Как итог, в других языках программирования есть типы, полностью совпадающие 
по структуре с signle и double (float и long float в C/C++, насколько я понимаю), и могут быть 
типы, полностью совпадающие с extended (типа long long float в некоторых компиляторах C/C++). 
Это, правда, нам будет фактически не  
важно, оно лишь приводит к тому, что, передавая, например, число типа single напрямую как оно записано в 
памяти "--- четырьмя байтами "--- в другую программу, вы без проблем сможете работать с ним там, 
даже если другая программа написана на другом языке, а вот с real не все так просто, надо понимать, какой именно это тип , "--- но такая 
необходимость в олимпиадных задачах возникает очень редко.
Аналогично, \texttt{file of single} (см. ниже) можно читать программой, написанной на любом (ну... 
почти любом "--- ещё бывают проблемы с порядком байт) языке программирования, а с \texttt{file of 
real} все сложнее.

Но все равно даже сам факт того, что real это либо single, либо double, либо что-то еще, делает тип real некоторым "<искусственным 
включением"> в системы типов. Это неудивительно: насколько я понимаю, во времена создания паскаля 
далеко не все компьютеры оснащались сопроцессором (да и, возможно, не было общепринятых стандартов 
хранения вещественных чисел в памяти), и потому был изобретён новый тип, чтобы можно было работать 
с вещественными числами и без сопроцессора. В современных же условиях нет необходимости так 
хитрить, и потому тип real фактически не нужен, поэтому его и приравняли к single/double. Поэтому 
фактически \textbf{никогда} не следует использовать в своих программах этот тип, пользуйтесь лучше single, double или extended. 

Какой же из трёх оставшихся типов использовать? Конечно, все зависит от задачи. Операции с single 
выполняются довольно быстро (но тем не менее, конечно, много дольше, чем операции с целыми 
числами), но, как правило, точности single не хватает в олимпиадных задачах (хотя в неолимпиадных 
вполне может хватать). Double обычно 
обеспечивает достаточную точность, но работает дольше, чем single. Extended обеспечивает ещё 
б\'{о}льшую точность, но ценой несоразмерного увеличения времени обработки. (Ну и, конечно, типы 
расходуют разное число байт: single 4, double 8, extended 10.) Тем не менее я всегда 
использовал где можно extended, и только если начинались проблемы с памятью или временем, переходил 
на double или в крайнем случае single. ИМХО, как правило, в олимпиадных задачах, где требуется 
вещественная арифметика, обычно время и память не столь критичны, чтобы надо было от extended 
уходить к другим типам.

Кстати, не забываем, что при работе с вещественными числами нельзя писать \verb|if a=0| и т.п., а надо работать с \verb|eps|.

\textbf{Нетривиальные целочисленные типы.} Пожалуй, ограничусь просто перечислением.
\begin{ulist}
\item \texttt{integer}. Казалось бы, ничего особенного, но в FP он занимает 2 или 4 байта в 
зависимости от настроек, и хранит соответствующий диапазон чисел. Советую в FP всегда писать 
\verb|{$mode delphi}| (см. ниже), и не иметь проблем. Двухбайтный тип, 
аналогичный BP'шному integer, называется smallint (а есть ещё и однобайтный shortint, такой же, как 
и в BP). 
\item \texttt{cardinal}, он же \texttt{longword}\footnote{Тут, похоже, есть какая"=то очень тонкая 
разница, связанная с поведение на разных платформах и ОС, но я её не знаю}. Это беззнаковый 
32-битный тип (т.е. как word относится к integer, так cardinal относится к longint). 
Соответственно, позволяет хранить числа от 0 до $2^{32}-1\approx 4\,000\,000\,000$.
\item \texttt{int64}. Знаковый (т.е. может хранить отрицательные числа) 64-битный тип. 
Хранит числа от $\approx -2^{63}$ до $\approx 2^{63}$ (пишу 
$\approx$, т.к. там где"=то граница отличается на единицу от степени двойки, даже понятно, где). 
\item \texttt{comp}. Аналогично int64 знаковый 64-битный тип. Но, в отличие от int64, он, видимо, 
поддерживается \textit{не} самим процессором, а сопроцессором (могу и ошибаться!). Поэтому есть ряд 
особенностей: во"=первых, comp был доступен даже в BP, во"=вторых, 
компилятор считает его вещественным типом со всеми вытекающими отсюда последствиями. В т.ч. его 
нельзя напрямую присваивать переменным целого типа, надо делать \texttt{round} и т.п.; с ним не 
работает \texttt{mod} и \texttt{div}, зато работает \texttt{/}, которое "--- внимание! "--- 
округляет результат по правилу round: 5/3=2, если считать в comp; при выводе в файл или на экран 
надо писать \verb|:0:0|, иначе будет вывод в виде с плавающей точкой (\verb|2.36348e5| типа того); 
кроме того, при выводе могут произойти глюки: существует небольшое количество значений comp, 
которые даже с \verb|:0:0| выводятся с ошибкой в паре последних знаков (хотя в памяти хранится 
точное значение!), поэтому с ним надо быть всегда осторожным. Т.е. comp выполняет все вычисления в 
пределах 64-битных целых значений абсолютно точно (как и любой целочисленный тип в пределах своей 
области значений), но есть приколы, которые надо понимать. Поэтому не используйте comp, используйте 
int64!
\end{ulist}